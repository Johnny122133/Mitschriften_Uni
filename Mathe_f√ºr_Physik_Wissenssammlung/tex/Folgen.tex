\textbf{Folge} in M ist eine Funktion von $\mathbb{N}_0$ nach M $f:\mathbb{N}_0 \to M, n \mapsto f(n)$\\

\textbf{beschränkte Folge:} $(a_n)_n$ ist beschränkt wenn $\exists b,c : \forall n: b<a_n<c$.

\textbf{nach oben beschränkte Folge:} $(a_n)_n$ ist nach oben beschränkt wenn $\exists b: \forall n: a_n<b$.

\textbf{nach unten beschränkte Folge:} $(a_n)_n$ ist nach unten beschränkt wenn $\exists b: \forall n: b<a_n$.\\

\subsection{Konvergenz von Folgen}
Eine Folge konvergiert, wenn der Grenzwert $\lim\limits_{n\to \infty} a_n$ existiert. Sie konvergiert gegen genau diesen Grenzwert.\\

\textbf{Einquetschkriterium} konvergieren $(a_n)_n$ und $(c_n)_n$ gegen den selben Wert $d$ und gilt $\forall n: (a_n)_n \leq (b_n)_n \leq (c_n)_n$ dann konvergiert auch $(b_n)_n$ gegen $d$.\\

\textbf{konvergente Folgen}
\begin{itemize}
    \item $\lim\limits_{n\to \infty} \sqrt[n]{a}=1$ für $a\in \mathbb{R}^+$
    \item $\lim\limits_{n\to \infty} \sqrt[n]{n}=1$
    \item $\lim\limits_{n\to \infty} (p)^n = 0$ konvergiert für $q<1$ gegen null, divergiert für $q>1$.
\end{itemize}

Eine momoton steigenden, nach oben beschränkten Folge konvergiert und es gilt $\lim\limits_{n\to \infty} a_n = \sup\{a_n:n\in \mathbb{N}_0\}$. Bei fallend inf.


