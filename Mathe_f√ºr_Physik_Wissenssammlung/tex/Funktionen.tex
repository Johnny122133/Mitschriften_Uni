
\subsection{Funktionen}
\subsubsection{surjektivität:}\label{subs:surjektivitaet}
wenn es für jedes $y$ aus $Y$ \textbf{mindestens} ein $x \in X$ mit $f(x) = y$ gibt.\\
wenn es auf jeder gedachten Horizontalen in der Zielmenge \textbf{mindestens} einen Schnittpunkt mit der Funktion gibt.\\
$\forall y \in Y : \exists \  x \in X:f(x)=y$

\subsubsection{injektivität}\label{subs:injektivitaet}
wenn es für jedes $y \in $ vom Wertebereich $Y$ \textbf{höchstens} ein $x \in$ der Definitionsmenge $X$ gibt.\\
Für jede gedachte Horizontale gibt es \textbf{höchstens} einen Schnittpunkt mit der Funktion.\\
$\forall x_1,x_2 \in X:f(x_1)=f(x_2) \Rightarrow x_1=x_2$\\
$\forall b \in B: \exists$ höchstens ein $a \in A :f(a)=b$\\

\subsubsection{bijektivität}\label{subs:bijektivitaet}  
\textbf{Injektiv und surjektiv}\\
wenn es für jedes $y\in Y$  \textbf{genau ein} $x \in X$ gibt.\\
wenn es auf jeder gedachten Horizontalen in der Zielmenge \textbf{genau einen Schnittpunkt} mit der Funktion gibt.\\
$\forall y \in Y : \exists ! \  x \in X:f(x)=y$