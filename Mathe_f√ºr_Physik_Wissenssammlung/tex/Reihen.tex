\begin{description}
    \item[\textbf{Definition:}] Eine Reihe ist eine Partialsumme einer Folge\\
    $(p_n)_n = \left(\sum\limits_{i=0}^{n}a_i \right)_n$, $p_n$... Reihe, $a_i$... Folge\\

    \item [\textbf{geometrische Reihe:}] $\sum\limits_{i=0}^{\infty} q^i$ mit $|q|<1$ konvergent, $\bm{\sum\limits_{i=0}^{\infty} q^i = \frac{1}{1-q}}$
    \item[\textbf{harmonische Reihe:}] $\sum\limits_{k=1}^{\infty}\frac{1}{k}$ ist divergent\\
    \item[\textbf{... Reihe}] $\sum\limits_{k=1}^{\infty}\frac{1}{k^r}$ konvergiert für $r\geq 2$\\
    \item[\textbf{Rechenregeln für konvergente Reihen:}]\ \\
    $\sum\limits_{i=0}^{\infty}(a_i+b_i) = \sum\limits_{i=0}^{\infty}a_i + \sum\limits_{i=0}^{\infty} b_i$\\

    $\sum\limits_{i=0}^{\infty}\gamma a_i = \gamma \sum\limits_{i=0}^{\infty} a_i$\\          
\end{description}



\subsection{Konvergenzkriterien für Reihen}

\textbf{absolute Konvergenz:} Eine Reihe $\sum_{i=0}^{\infty}a_i$ in $\mathbb{R}$ oder $\mathbb{C}$ heißt absolut konvergent, falls $\sum_{i=0}^{\infty}|a_i|$ konvergiert. Es gilt:$|\sum_{i=0}^{\infty}a_i| \leq \sum_{i=0}^{\infty}|a_i|$.\\

Eine \textbf{Komplexe Reihe} $\sum_{i=0}^{\infty}a_i$ konvergiert/konvergiert absolut wenn $\sum_{i=0}^{\infty}\text{Re}(a_i)$ und $\sum_{i=0}^{\infty}\text{Im}(a_i)$ konvergieren/absolut konvergieren.\\

\textbf{konvergiert die Reihe} $\sum_{i=0}^{\infty}a_i$ muss ($\lim_{i\to\infty}a_i = 0$) sein (notwendiges aber nicht hinreichendes Kriterium).\\

\textbf{$\bm{\sum_{k=0}^{\infty}a_k}$ konvergiert} genau dann, wenn die Folge von \textbf{Partialsummen} beschränkt ist.

\subsubsection{Leibnitz Kriterium}
Sei $(a_k)_k$ eine monoton fallende Nullfolge positiver reeller Zahlen dann konvergiert\\
\[\sum\limits_{k=0}^{\infty}(-1)^ka_k\]
zeigt Konvergenz aber \textbf{nicht} absolute Konvergenz.


\subsubsection{Minoranten- Majoranten Kriterium}
falls $\forall n: 0\leq a_n \leq b_n \wedge \sum_{i=0}^{\infty}b_i$ konvergent, dann konvergiert auch $\sum_{i=0}^{\infty}a_i$\\

falls $\forall n: 0\leq a_n \leq b_n \wedge \sum_{i=0}^{\infty}a_i$ divergent, dann divergiert auch $\sum_{i=0}^{\infty}b_i$\\


\subsubsection{Quotientenkriterium}
Sei $(a_n)_n$ eine Folge in $\mathbb{R}$ oder $\mathbb{C}$ und $\forall n :a_n \neq 0$ dann ist $\sum_{i=0}^{\infty}a_i$ absolut konvergent wenn:
\[\lim\limits_{n\to \infty}\sup\left|\frac{a_{n+1}}{a_n}\right|<1\]
divergent wenn obiges $>1$ ist.

\subsubsection{Wurzelkriterium}
Sei $(a_n)_n$ eine beschränkte Folge in $\mathbb{R}$ oder $\mathbb{C}$ dann ist $\sum_{i=0}^{\infty}a_i$ absolut konvergent wenn:
\[\lim\limits_{n\to \infty}\sup \sqrt[n]{|a_n|}<1\]
divergent wenn obiges $>1$.

