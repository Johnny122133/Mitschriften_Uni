\begin{center}
    \begin{tblr}{ll}
        \textbf{Symbol}         & \textbf{Bedeutung} \\ \hline[1.5pt]
         $R$                    & Relation \\\hline
         $\in/ \notin$          & Element / kein Element von \\\hline
         $\forall / \nexists$   & für alle / für kein \\\hline
         $\exists$              & Existenzquantor, mindestens ein \\\hline
         $\exists !$            & Anzahlquantor, genau ein \\\hline
         $A \subset B$          & echte Teilmenge, $a \in A \land a,b \in B:\exists b\notin A$\\ \hline
         $A \subseteq B$        & Teilmenge $a \in A \land a,b \in B$ \\\hline
         $]1,3[$, $(1,3)$       & $1<x<3$ \\\hline
         $[1,3]$                & $1\leq x \leq 3$ \\\hline
         $\Rightarrow$          & genau dann wenn \\\hline
         $\Leftrightarrow$      & aus Aussage A folg B und umgekehrt \\\hline
         $\rightarrow$          & Abbildungsvorschrift für Mengen \\\hline
         $\mapsto$              & Abbildungsvorschrift für Elemente \\\hline
         $\circ$                & Komposition / Verkettung von Funktionen \\\hline
         $\land\ /\ \lor$       & und / oder \\\hline
         Lemma                  & Hilfssatz \\\hline
         $\overline{z}, \ z^*$  & konjungiert komplexe Zahl \\\hline
         $\preccurlyeq$         & beliebiges Symbol \\\hline
         $\stackrel{?}{=}$      & zu zeigen \\\hline
         $\stackrel{!}{=}$      & soll erfüllt sein um ... zu zeigen \\\hline
         $:=, \equiv$           & definiere \\\hline
         $\cup,\cap, /$         & Vereinigung, Durchschnitt, Subtrahiert \\\hline
         disjunkt               & $A \cap B = \{\}$ \\\hline
         infimum \\\hline
         supremum \\\hline

    \end{tblr}
\end{center}


