\begin{center}
    \begin{longtblr}[
        caption = {Grundbegriffe},
        label = {tblr:Grundbegriffe}
    ]{
        colspec = {l l},
        rowhead = 1,
        row{odd} = {gray9}
    }
        \textbf{Symbol}         & \textbf{Bedeutung} \\ \hline[1.5pt]
        $R$                    & Relation \\\hline
        $\in/ \notin$          & Element / kein Element von \\\hline
        $\forall / \nexists$    & für alle / für kein \\\hline
        $\exists$               & Existenzquantor, mindestens ein \\\hline
        $\exists !$             & Anzahlquantor, genau ein \\\hline
        $A \subset B$           & echte Teilmenge, $a \in A \land a,b \in B:\exists b\notin A$\\ \hline
        $A \subseteq B$              & Teilmenge $a \in A \land a,b \in B$ \\\hline
        $\{1,3\}$           &   $x={1,3}$, die Zahlen 1 und 3 \\ \hline
        $]1,3[$, $(1,3)$       & $1<x<3$ \\\hline
        $[1,3]$                & $1\leq x \leq 3$ \\\hline
        $\Rightarrow$          & genau dann wenn \\\hline
        $\Leftrightarrow$      & aus Aussage A folg B und umgekehrt \\\hline
        $\rightarrow$          & Abbildungsvorschrift für Mengen \\\hline
        $\mapsto$              & Abbildungsvorschrift für Elemente \\\hline
        $\circ$                & Komposition / Verkettung von Funktionen \\\hline
        $\land\ /\ \lor$       & und / oder \\\hline
        Lemma                  & Hilfssatz \\\hline
        $\overline{z}, \ z^*$  & konjungiert komplexe Zahl \\\hline
        $\preccurlyeq$         & beliebiges Symbol \\\hline
        $\stackrel{?}{=}$      & zu zeigen \\\hline
        $\stackrel{!}{=}$      & soll erfüllt sein um ... zu zeigen \\\hline
        $:=, \equiv$           & definiere \\\hline
        $\cup,\cap, \backslash$         & Vereinigung, Durchschnitt, Subtrahiert \\\hline
        disjunkt               & $A \cap B = \{\}$ \\\hline
        infimum \\\hline
        supremum \\\hline
        notwendiges Kriterium & muss immer erfüllt sein, reicht aber nicht aus\\ \hline
        hinreichendes Kriterium & wenn erfüllt dann ... \\ \hline
        Nullfolge  &   $(a_k)_k$ ist eine Nullfolge wenn $\lim_{k\to\infty}a_k=0$\\ \hline
        surjektiv & $\forall y \in Y : \exists \  x \in X:f(x)=y$ \ref{subs:surjektivitaet}\\ \hline
        injektiv &$\forall x_1,x_2 \in X:f(x_1)=f(x_2) \Rightarrow x_1=x_2$   \ref{subs:injektivitaet}\\ \hline
        bijektiv & $\forall y \in Y : \exists ! \  x \in X:f(x)=y$ \ref{subs:bijektivitaet}\\ \hline
        beschränkte Folge & $\exists b,c : \forall n: b<a_n<c$ \\ \hline
        nach oben beschränkte Folge &  $\exists b: \forall n: a_n<b$\\ \hline
        nach unten beschränkte Folge &  $\exists b: \forall n: b<a_n$\\ \hline
        $\sum\limits_{i=1}^{n}i=\frac{n(n+1)}{2}$ & \\ \hline
    \end{longtblr}
\end{center}

\textbf{Zwischenwertsatz:} \\
Eine im Intervall $[a,b]$ stetige Funktion nimmt jeden Wert zwischen $f(a)$ und $f(a)$ mindestens einmal an.\\

\textbf{Satz von Rolle:}\\
Es sei $u,v\in \mathbb{R}$ mit $u<v$ und $f:[u,v] \to \mathbb{R}$ eine differenzierbare Funktion mit $f(u) = f(v)$. fann gibt es ein $x \in ]u,v[ \to \mathbb{R}$ mit $f'(x)=0$.\\

\textbf{Mittelwertsatz der Differentialrechnung:}\\
Es sei $u,v\in \mathbb{R}$ mit $u<v$ und $f:[u,v] \to \mathbb{R}$ eine differenzierbare Funktion. Dann gibt es ein $x \in ]u,v[$ mit $f'(x)= \frac{f(v)-f(u)}{v-u}$.\\

\textbf{veralgemeinerter Mittelwertsatz der Differentialrechnung:}\\
Es sei $u,v\in \mathbb{R}$ mit $u<v$ und $f,g:[u,v] \to \mathbb{R}$ eine differenzierbare Funktionen und $\forall x: g'(x) \neq 0$. Dann gibt es ein $x \in ]u,v[$ mit $\frac{f'(x)}{g'(x)} = \frac{f(v) - f(u)}{g(v)-g(u)}$.\\



