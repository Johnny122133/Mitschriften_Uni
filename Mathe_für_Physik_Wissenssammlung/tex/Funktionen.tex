
\subsubsection{surjektivität:}\label{subs:surjektivitaet}
$\bm{\forall y \in Y : \exists \  x \in X:f(x)=y}$\\
wenn es für jedes $y$ aus $Y$ \textbf{mindestens} ein $x \in X$ mit $f(x) = y$ gibt.\\
%wenn es auf jeder gedachten Horizontalen in der Zielmenge \textbf{mindestens} einen Schnittpunkt mit der Funktion gibt.\\

\subsubsection{injektivität}\label{subs:injektivitaet}
$\bm{\forall x_1,x_2 \in X:f(x_1)=f(x_2) \Rightarrow x_1=x_2}$\\
$\forall b \in B: \exists$ höchstens ein $a \in A :f(a)=b$\\
wenn es für jedes $y \in $ vom Wertebereich $Y$ \textbf{höchstens} ein $x \in$ der Definitionsmenge $X$ gibt.\\
%Für jede gedachte Horizontale gibt es \textbf{höchstens} einen Schnittpunkt mit der Funktion.\\

\subsubsection{bijektivität}\label{subs:bijektivitaet}  
$\bm{\forall y \in Y : \exists ! \  x \in X:f(x)=y}$\\
\textbf{Injektiv und surjektiv}\\
wenn es für jedes $y\in Y$  \textbf{genau ein} $x \in X$ gibt.\\
%wenn es auf jeder gedachten Horizontalen in der Zielmenge \textbf{genau einen Schnittpunkt} mit der Funktion gibt.\\

\subsection{Relationen}
\textbf{Definition Relation:} Es seien $A_1, ..., A_n, n\geq 1$ Mengen, dann heißt die Teilmenge von $R \subseteq A_1 \times ... \times A_n$ eine Relation zwischen den Mengen $A_1,...,A_n$.\\
\textbf{zweistellige Relation:} n=2\\

\textbf{binäre Relation:} $A=A_1=A_2$ dann wird meist $a_1Ra_2$ oder $R(a_1,a_2)$ geschrieben.\\

\textbf{Typen von Relationen:}
\begin{enumerate}
    \item R heißt \textbf{reflexiv}, falls $aRa$ für alle $a\in A$ gilt.
    \item R heißt \textbf{irreflexiv}, falls $aRa$ für kein $a\in A$ gilt.
    \item R heißt \textbf{symmetrisch}, falls für $a,b \in A\ aRb$ genau dann gilt, wenn $bRa$.
    \item R heißt \textbf{antisymmetrisch}, falls für $a,b \in A$ aus $aRb$ und $bRa$ stets $a=b$ folgt.
    \item R heißt \textbf{transitiv}, falls für $a,b,c \in A $ aus $aRb$ und $bRc$ stets $aRc$ folgt.
\end{enumerate}

\textbf{Komposition von Relationen:} (Hintereinanderausführung) $S\circ R = \{(a,c) \in A\times C : \text{ es gibt ein }b\in B \text{ mit } aRb \text{ und } bSc \}$ (mit $R\subseteq A\times B$ und $S \subseteq B\times C$ zweistellingen Relationen)\\

\textbf{inverse Relation:} $R^{-1}= \{(b,a) \in B\times A:(a,b)\in R\}$\\
${(R^{-1})}^{-1} = R$, $(S \circ R)^{-1} = R^{-1} \circ S^{-1}$\\



\subsection{Funktionen}

\textbf{Definition:} $\{\forall a \in A: \exists ! b\in B : \text{ mit } (a,b)\in f\}$\\

\textbf{Definition 1.4.1}
\begin{enumerate}
    \item Die Menge $f(A) = \{f(a \in A : f(a))$
\end{enumerate}

Bei stetigen Funktionen darf der Grenzwert und die Funktion vertauscht werten ($\lim\limits_{x\to 0 }e^{x\cdot \ln (x)}=e^{\lim\limits_{x\to 0 }x\cdot \ln(x)}$).