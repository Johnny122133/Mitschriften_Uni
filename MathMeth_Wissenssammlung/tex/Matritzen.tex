
\textbf{lineare Abhängig:} Eine Zeile einer Matrix ist das Vielfache einer Anderen.\\

\subsection{Determinante}
\[A=\begin{pmatrix}
 a_1 & b_1 \\
 a_2 & b_2 \\
\end{pmatrix}\qquad \text{det}(A)=a_1 \cdot b_2 - a_2 \cdot b_1\]
\ \\
\[A=\begin{pmatrix}
 a_1 & b_1 & c_1 \\
 a_2 & b_2 & c_2 \\
 a_3 & b_3 & c_3 \\
\end{pmatrix}
\begin{matrix}
 a_1 & b_1 \\
 a_2 & b_2 \\
 a_3 & b_3 \\
\end{matrix}\]
\ \\
\[\text{det}(A)= + a_1 \cdot b_2 \cdot c_3 + b_1 \cdot c_2 \cdot a_3 + c_1 \cdot a_2 \cdot b_3 - a_3 \cdot b_2 \cdot c_1 - b_3 \cdot c_2 \cdot a_1 - c_3 \cdot a_2 \cdot b_1\]
\ \\
\textbf{Regel von Sarrus:} Diagonalen miteinander multiplizieren und anschließend summieren. Geht nur bei 3x3 Matrix.\\

\textbf{Rechenregeln zur Determinante:}\\
$z_1,...,z_n$ ... Zeilen, $s_1,...,s_n$... Spalten der Matrix
\begin{itemize}
    \item $\det(s_1,s_2,s_3) = (-1)\cdot \det (s_1,s_3,s_2) = \det(s_3,s_1,s_2)$\\
    werden zwei Zeilen/Spalten vertauscht, muss die Determinante mit $(-1)$ multipliziert werden.
    \item Addition des Vielfachen einer zeile/Spalte zu einer Anderen ändert den Wert nicht.
    \item $\det(k\cdot z_1,z_2,z_3)^T = k \cdot \det(z_1,z_2,z_3)^T \qquad \det(k\cdot a) = k^3\cdot  \det(A)$
    \item $\det(A^{-1} = \frac{1}{\det(A)})$
    \item $\det(A^k) = (\det(A))^k$
    \item $\det(A\cdot B) = \det (A)\cdot \det(B)$
    \item Beil linearer Abhängigkeit ist die Determinante 0
    \item $\bm{\vec{a}\cdot (\vec{b}\times \vec{c})=\det(\vec{a}, \vec{b}, \vec{c})}$
\end{itemize}

\subsubsection{Laplace´scher Entwicklungssatz:}

\[
\text{det}\ A=\sum\limits_{j=1}^{n}(-1)^{i+j}\cdot a_{ij}\cdot \text{det}\ A'_{ij}\qquad \text{für ein festes}\ i\text{ oder } j \in \{1,\ldots,n \}
\]

Die Determinante ändert sich nicht, wenn zu einer Zeile/Spalte ien Vielfaches einer anderen Zeile/Spalte hinzu addiert. Mit dieser Methode versucht man möglichst viele Nullen in eine Zeile/Spalte, nach der man dann die Determinante entwickelt, bekommt.\\

\textbf{Gausß-Jordan-Verfahren:} Beschreibt die elementare Zeilenumformung zum Vereinfachen der Matrix um das Berechnen der Determinante zu erleichtern. Dazu dürfen ganze Zeilen oder Spalten mit einem Faktor multipliziert werden. Weiters dürfen Vielfache von Zeilen/Spalten zu anderen Zeilen/Spalten addiert werden. Wenn zwei Zeilen/Spalten vertauscht werden, ändert sich das Vorzeichen der Determinante.

\subsection{Matrix Multiplikation}

\[A=\begin{pmatrix}
 a_1 & b_1 \\
 a_2 & b_2 \\
\end{pmatrix} \qquad
B=\begin{pmatrix}
 c_1 & d_1 \\
 c_2 & d_2 \\
\end{pmatrix}\]

\[A\cdot B = 
A=\begin{pmatrix}
 a_1 \cdot c_1 + b_1 \cdot c_2 & a_1 \cdot d_1 + b_1 \cdot d_2 \\
 a_2 \cdot c_1 + b_2 \cdot c_2 & a_2 \cdot d_1 + b_2 \cdot d_2 \\
\end{pmatrix}\]\\

\subsection{Eigenwert und Eigenvektor}
charakteristisches Polynom: $P(\lambda)=\text{det}(A-\lambda I)$\\

Skript 2022x12x07 DGL Systeme Seite 4\\

\textbf{Eigenwerte:} Nullstellen $\lambda_i$ von $P(\lambda)$, $P(\lambda)=\text{det}(A-\lambda I) \stackrel{!}{=} 0$\\

\textbf{Eigenvektor:} $\text{Eig}(A, \lambda_i):(A-\lambda_i I)\cdot \vec{x} = 0$
mit $\vec{x} = \begin{pmatrix}
    x_1\\
    x_2\\
    x_3\\
\end{pmatrix}$\\
Die Gleichung nach $x_1,x_2,...1$ auflösen, dann kann der Eigenvektor in Form \\
\[\text{Eig}(A,\lambda_i)= \left\{ \vec{x} \in \mathbb{R}^3 : \vec{x} = r \cdot
\begin{pmatrix}
    x\\
    y\\
    z\\
\end{pmatrix} ; x \in \mathbb{R} \right\}\]
geschrieben werden.

 % \[\text{Eig}(A,\lambda_i)= \left { \vec{x} \in \mathbb{R}^3 : \vec{x} = r \cdot
% \begin{pmatrix}
%     x\\
%     y\\
%     z\\
% \end{pmatrix} ; x \in \mathbb{R} \right } \]


\subsection{Invertieren einer Matrix}
\begin{boxedminipage}{\textwidth}
    \textbf{Invertieren einer Matrix mit Gauß-Jordan-Verfahren}
    \begin{itemize}
        \item Eine Matrix ist invertierbar, wenn sie
        \item - quadratisch ist
        \item - und die Determinante $\neq$ 0 ist.
    \end{itemize}
\end{boxedminipage}
\ \\

Neben die Matrix wird die Einheitsmatrix geschrieben

\[\left (
    \begin{array}{rrr}
        a_1 & b_1 & c_1 \\
        a_2 & b_2 & c_2 \\
        a_3 & b_3 & c_3 \\
    \end{array}
    \right .
    \left |
    \begin{array}{rrr}
        1 & 0 & 0 \\ 
        0 & 1 & 0 \\
        0 & 0 & 1 \\ 
    \end{array} \right )\]

Die Matrix wird solange mit elementarer Zeilenumformung umgeformt, bis auf der linken Seite die Einheitsmatrix steht (Gauß-Jordan-Verfahren). \\

\textbf{elementare Zeilenumforumengen:}  
\begin{itemize}
    \item Vertauschen von zwei Zeilen
    \item Multiplikation einer Zeile mit $k \neq 0$
    \item Addition des Vielfachen eriner Zeile zu einer anderen
\end{itemize}



\subsection{Wichtige Identitäten}
\begin{center}
    \begin{tblr}{c}
        Identität \\ \hline[1.5pt]
        det $(A^{-1})= \frac{1}{\text{det}(A)}$\\ \hline
        $(A^{-1})^{-1} = A$ \\ \hline
        $(A\cdot B)^{-1}=B^{-1}\cdot A^{-1}$ \\ \hline 
        $(A^t)^{-1}=(A^{-1})^t$ \\ \hline
        $A\cdot A^{-1}=A^{-1}\cdot A=I_n$ \\ \hline 
        $\vec{a}\cdot (\vec{b}\times \vec{c})=\det(\vec{a}, \vec{b}, \vec{c})$\\ \hline
    \end{tblr}
\end{center}



\subsection{Lösungsverfahren:}
\subsubsection{Gaußsches Eliminationsverfahren:}

$\mathbf{A}\cdot \vec{x} =\vec{b}\qquad\rightarrow \qquad \left (\mathbf{A} \left | \vec{b}\right ) \right .$

Soll $\vec{x}$ berechnet werden, wird eine Matrix der Form $(\mathbf{A}|\vec{b})$ aufgeschrieben und solange mit elementarer Zeilenumformung umgeformt, bis über oder unter der Hauptdiagonale von A nur nullen stehen. Mit dem wissen dass die n-te Spalte der Matrix mit $x_n$ multipliziert wird, kann in die erste/letzte Zeile eingesetzt werden.


\subsubsection{Cramersche Regel:}
Sei $A=(a_{ij})$ eine $n\times n$ - Matrix mit den Spaltenvektoren $(a^1,a^2,...,a^n)$ und $\text{det}A\neq 0$. Dann ist die Lösung des LGS
\[a\cdot x = b \qquad \text{mit} \qquad x = (x_1, ... , x_n)^T, b=(b_1,...,b_n)^T\]
gegeben durch
\[x_i=\frac{\text{det}(a^1,...a^{i-1}, b, a^{i+1},...,a^n)}{\text{det}A}\]

Man ersetz die $i$-te Spalete von $A$ durch die Lösung $b$ des LGS. Dann ist $x_i$ der Quotient der so entstandenen Matrix und det$A$.