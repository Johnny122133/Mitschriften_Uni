
\subsection{Fourrierreihe}
\textbf{gerade Funktionen:} können nur durch \textbf{gerade} Funktionen dargestellt werden\\
$b_n=0:\forall k \in \mathbb{N}$  

\textbf{ungerade Funktionen:} können nur durch \textbf{ungerade} Funktionen dargestellt werden\\
$a_n=0:\forall k \in \mathbb{N}$  

\textbf{weder gerade noch unerade:} werden durch eine Kombination aus geraden und ungeraden Funktionen dargestellt.\\
$b_n, a_n\neq0$ 

\subsection{Taylorreihe}

\[T(f(x))=\sum\limits_{0}^{n} \frac{f^n(x)}{n!}\cdot {(x-x_0)}^n\]
mit $f^n$ ist di n-te Ableitung von f