\textbf{Normaleinheitsvektor:} $N = \frac{\nabla\phi(r)}{|\nabla\Phi(r)|}$\\
\textbf{Nabla:} $\bm {\nabla} = \left(\frac{\partial}{\partial x_1}, \frac{\partial}{\partial x_2}, \frac{\partial}{\partial x_3} \right)$

\begin{center}
    \begin{tblr}{| l | l |}
        \hline
        Bedeutung & Berechnung\\ \hline[1.5pt]
        grad$\Phi = \left(\frac{\partial \Phi}{\partial x_1}, \frac{\partial \Phi}{\partial x_2}, \frac{\partial \Phi}{\partial x_3} \right)$ & $\mathbf{\nabla}\Phi$ \\ \hline
        div$\mathbf{A}=\mathbf{\nabla \cdot A} = \mathbf{\nabla \cdot \nabla}\Phi$ & $\mathbf{\nabla\cdot A}$\\ \hline
        rot$\mathbf{A}$ = rot grad$\Phi$ = $\mathbf{\nabla \times \nabla} \Phi$ & $\mathbf{\nabla \times A}$\\ \hline
    \end{tblr}
\end{center}

\textbf{Gradient}\\
Erzeugt ein Vektorfeld mit den Steigungen des skalaren Feldes aus einem skalaren Feld. (Vektor)\\

\textbf{Divergenz}\\
Gibt an ob die Vektoren in diesem Punkt zusammen oder auseinander zeigen. (Skalar)

\begin{itemize}
    \item div$\mathbf{A} > 0A>0$: im Volumenelement befindet sich eine Quelle (mehr raus als rein)
    \item div$\mathbf{A} < 0A<0$: im Volumenelement befindet sich eine Senke (mehr rein als raus)
    \item div$\mathbf{A} = 0A=0$: Im Volumenelement befindet sich weder eine Quelle noch eine Senke, das Vektorfeld ist an dieser Stelle quellenfrei.
\end{itemize}

\textbf{Rotation}\\
Gibt an, ob sich das Vektorfeld um einen Punkt dreht. (Vektor)