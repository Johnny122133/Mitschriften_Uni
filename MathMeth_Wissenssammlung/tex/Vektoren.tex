\begin{description}
    \item [\textbf{Einheitsvektor:}] Jeder Vector mit Betrag 1. $\vec{e_x}=\frac{\vec{x}}{|\vec{x}|}$\\

    \item[\textbf{Skalarprodukt:}] $\vec x \cdot \vec{y} = x_1\cdot y_1 + x_2\cdot y_2 + x_3\cdot y_3 = |\vec{x}|\cdot|\vec{y}|\cdot \cos(\varphi)$\\
    Null, wenn $\vec x $ und $\vec y $ normal aufeinander stehen. Gibt die gemeinsame Länge von in eine  Richtung an.\\
    \textbf{Schwarzsche Ungleichung:} $|x\cdot y|\leq |x|\cdot |y|$\\
    
    \item[\textbf{Kreuzprodukt:}] $z = \vec x \times \vec y  = 
    \begin{pmatrix}
        x_2y_3-x_3y_2\\
        x_3y_1-x_1y_3\\
        x_1y_2-x_2y_1\\
    \end{pmatrix}$\\
    
    $z$ steht \textbf{senkrecht} auf $\vec x $ und $\vec y $\\
    $A = |\vec x \times \vec y |$... Fläche des von $\vec x $ und $\vec y $ aufgespannten Parallelogramms.\\
    Kreuzt man einen Vektor mit sich selbst ergibt das immer \textbf{null}.\\
    Dies ist spätestends bei $\nabla \times \nabla = 0$ gut zu wissen.\\

    \item[\textbf{Winkel zwischen Vektoren:}]  $\varphi = \arccos \frac{\vec x \cdot \vec y }{|\vec{\vec x }|\cdot |\vec y |}$\\
    
    \item[\textbf{Spatprodukt:} ] Volumen welches drei Vectoren spannen, $V = |\vec a \times \vec b |\cdot |\vec c | \cdot \cos \varphi = (\vec a \times \vec b )\cdot \vec c $\\
    
    \item[\textbf{Projektion von x in Richtung y:}] $x_y=\vec{e_y}\cdot \vec{x} \cdot \vec{e_y}=\frac{\vec{y}\cdot \vec{x}}{|\vec{y}|^2} \cdot \vec{y}$\\
    $\vec{e_y}\cdot \vec{x} = |\vec{e_y}|$ gibt die Länge von x in Richtung y an. Multipliziert mit $\vec{e_y}$ um einen Vektor zu erhalten.
    

\end{description}
 