
Wenn die DGL linear ist, ist eine Kombination der Lösung durch Addition auch wieder eine Lösung.

Ansatz: $y_0=C\cdot e^{\gl \cdot x}$

\textbf{homogene Lösung} (Skript 2022x12x07 Seite 8)
\begin{itemize}
    \item Ansatz in DGL einsetzen. Dabei die DGL als homogen behandeln.
    \item $C\cdot e^{\gl \cdot x}$ herausheben. Der Rest bildet das charakteristische Polynom.Dieses 0 setzen.
    \item Alle Nullstellen eingesetzt in den Ansatz sind Lösungen der DGL. Die einzelnen Lösungen können linear kombiniert werden, sprich jede Addition von Lösungen ist wieder eine Lösung der DGL. Die \textbf{allgemeine homogene} Lösung ist die Summe der Lösungen, multipliziert mit Konstanten (nur bei linearen DGL). Durch geschicktes Kombinieren von Komplexe Lösungen können reelle konstruiert werden.
\end{itemize}

\textbf{partikuläre Lösung}
\begin{itemize}
    \item Partikuläre Lösung von f(x) mit Tabelle oder durch Variation der Konstanten bilden.
    \item Homogene und partikuläre Lösung addieren.    
\end{itemize}



