
\[ \bm{\dot{\vec{y}}(t)=A\cdot\vec{y}(t) + \vec{f}(t)}\]


\textbf{Kochrezept:}\\
\begin{itemize}
    \item Aufstellen der Systemmatrix ($y'(t) = A y(t)$)
    \item Eigenwerte $\lambda$ von A bestimmen ($\text{det}(A-\lambda I_n)\stackrel{!}{=}0$)
    \item Eigenvektoren von A berechnen ($(A-\lambda_i I_n)\vec{x} \stackrel{!}{=}\vec{0}$)
    \item Fundamentalsystem ($\vec{x}_i e^{\lambda_i t}$)
    \item allgemeine Lösung (Summe des Fundamentalsysteme)
    \item homogene Lösung (Anfangsbedingungen einsetzen)
    \item partikuläre Lösung Variation der Konstanten oder Ansatz
\end{itemize}

\subsubsection{partikuläre Lösung:}
\textbf{Zeilenweise auflösen:}\\
Die partikuläre Lösung kann Zeilenweise über Variation der Konstanten oder per Ansatz mit Tabelle bestimmt werden.\\

Der partikuläre Ansatz wird in die Differentialgleichung Zeilenweise eingesetzt und anschließend in die Form
\begin{align*}
    B \begin{pmatrix}
        \dot{C}_1 e^{i\lambda_1 t}\\
        \vdots\\
        \dot{C}_ie^{i\lambda_i t}\\
    \end{pmatrix} = \vec{f(t)}  
\end{align*}

gebracht. Durch invertieren kann umgeformt werden:

\begin{align*}
    \begin{pmatrix}
        \dot{C}_1 e^{i\lambda_1 t}\\
        \vdots\\
        \dot{C}_ie^{i\lambda_i t}\\
    \end{pmatrix} =B^{-1} \vec{f(t)}  
\end{align*}

Die Komponenten von $\vec{C}(t)$ können durch zeilenweises umformen und integrieren bestimmt werden.