%%% Article Template
%%% Set Document layout
\documentclass[final, 12pt]{article}

%%%%%%%%%% Set Name, Subject, Date ...
\newcommand{\toppic}{Fehlerrechnung}
\newcommand{\mytitle}{Messstatistik und Fehlerfortpflanzung}
\newcommand{\workingDate}{11.01.2023}
\newcommand{\userName}{Jonathan Mayer}
%%
\usepackage{options}

%%%%%%%%%% Begin the document
\begin{document}  
\begin{center}
    {\textbf {\huge \titel}}\\[5mm] % titel
    {\large \userName} \\[5mm]      % User Name
    \workingDate\\                  % Working Date
\end{center} %%%%%%%%%% Titel %%%%%%%%%%%%%%

%%
\section{Fehler}
\subsection{Qunatisierungsfehler}
Der Qunatisierungsfehler beträgt ein halbes \textbf{Skalenteil}.\\

\subsection{Fehlerforpflanzung:}
$f$... Funktion, $u_{x_i}$ Standardabweichung/Unsicherheit
\begin{equation}
    \textbf{Gauß´sche Fehlerfortpflanzung:}\qquad u_z = \sqrt{\sum\limits_{i=1}^{N} \left|\frac{\partial f}{\partial x_i}\right|^2 \cdot u_{x_i}^2}   
    \label{eq:Gauß}    
\end{equation}

\section{Messtatistik}

\begin{equation}
    \text{Gewichteter Mittelw.:} \quad \mu = \frac{\summ{i=1}{N}(g_i*x_i)}{\summ{j=1}{N}g_j} \qquad g_i=\frac{1}{\sigma_{xi}^2}
    \label{eq:Mittelw}
\end{equation} 


\begin{equation}
    \text{innere Varianz: }\qquad \sigma_{\mu in}^2= \frac{1}{\summ{i=1}{N}\frac{1}{\sigma_{xi}^2}} \qquad \text{äußere Varianz: }\qquad \sigma_{\mu ex}^2= \frac{\summ {i=1}{N}\frac{(x_i-\mu)^2}{\sigma_{xi}^2}}{\summ {i=1}{N}\frac{1}{\sigma_{xi}^2}}
    \label{eq:Varianz}
\end{equation} 
       

%%
\section{Optik:}

\subsection{Bessl Verfahren:}

\section{Elektrik}

\subsection{Widerstände}
\textbf{Reihenschaltung:} $R_{ges} = \summ {0}{n} R_n$\\

\textbf{Parallelschaltung:} $R_{ges} = \frac{1}{\sum\limits_{n=0}^{N} \frac{1}{R_n}}$\\

\textbf{Knotenregel:} $\sum R_{in}=\sum R_{out}$\\
Die Summe der Ströme in den Knoten muss gleich der Summe der Ströme aus dem Knoten sein.
\\

\textbf{Maschenregel:} Summe aller Spannungn ist 0. 

\subsection{Statistik, lineare Regression}

$x^2=\sum\limits_{i}^{N=5} \left ( \frac{d_i}{s_i} \right )^2$ not finished

wenn $s_i =$ const.: $x^2 = \frac{1}{s_i^2}\sum d_i^2$

\textbf{Optimierungsbedingung für Parameter $a_0, a_1$:}

$d^2=\sum\limits_{i=1}^{N}\left(\frac{y_i-a_0-a_1x_i}{s_i}\right)^2 \rightarrow $ Minimum

--> Minimum durch Ableitung:\\
$0=\frac{dx^2}{da_0}=-2 \sum\left(\frac{y_i-a_0-a_1x}{s_i^2}\right)$\\
$0=\frac{dx^2}{da_1}=-2 \sum\left(\frac{y_i-a_0-a_1x}{s_i^2}\right)$


$<x>$... Erwartungswert

%%% Verzeichnisse
\newpage
%\listoftables
%\listoffigures
%\printbibliography{}
%%%%%% Document end
\end{document}     %%% End the document


