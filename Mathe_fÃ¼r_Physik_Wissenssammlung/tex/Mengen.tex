\textbf{explizite Angabe:} $A=\{1,2,3,4,...\}$\\

\textbf{spezifikation über charakteristische Eigenschften:} $B = \{a\in A:\varphi (a)\}$\\

$\{b\in A': \text{ es gibt ein } a \in A \text{ mit } b = t(a)\}$: $\{t(a): a \in A\}$\\

\textbf{Definition 1.1.4:}
\begin{enumerate}
    \item Zwie Mengen $A$ und $B$ sehen wir als gleich an, wenn sie dieselben Elemente enthalten.
    \item Eine Menge $A$ heißt Teilmenge einer Menge $B$ ($a \subseteq B$) falls $\forall a: a \in B$.\\
    Echte Teilmenge ($A \subset B$): $\{\exists b\in B: b \notin A\}$
    \item Die Menge aller Tielmengen einerm Menge $A$ heißt Potenzmenge von $A$ und wird mit $P(a)$ bezeichnet.
\end{enumerate}

\textbf{Definition 1.1.7:}
\begin{enumerate}
    \item \textbf{Durchschnitt:} $A \cap B = \{x \in U : x\in A \land x \in B\}$
    \item \textbf{Vereinigung:} $A \cup B = \{x \in U : x\in A \lor x \in B\}$
    \item \textbf{Differenz:} $A \backslash B = \{x \in U : x\in A \land x \notin B\}$
    \item \textbf{Komplement:} $\complement A = U\backslash A$
\end{enumerate}
\textbf{disjunkt:} $a \cap B  = \{\}$\\

\textbf{Kartesisches Produkt:} $A \times B = A^2 = \{(a,b):a\in A \text{ und } b \in B \}$\\
Die Menge aller geordneten Paare $(a,b)$ mit $a \in A$ und  $b \in B$.\\


