\subsection{Thermodynamik}

\textbf{Nullter Hauptsatz:} $A=C \wedge B=C \Rightarrow A=B$\\
Befinden sich zwei Systeme A und B jeweils bis Sytem C im Gleichgeweicht, befinden sich auch A und B im Gleichgewicht.\\

$\bm{p \cdot V = const.} \qquad \text{wenn } T=const$. \qquad (Boyle-Mariottesches Gesetz)\\

$\bm{V=V_0+\alpha \cdot T} \text{ für } p=const.$ \qquad (Gay Lussac)\\

1 Torr = 1 mm Hg, 1 bar = $10^5$ Pa [N/mm$^2$], Normaldruck: 1013 mbar = 1013 hPa = 760 Torr\\

Ideale Gasgleichung: $\bm{p\cdot V = N\cdot k_B \cdot T}$\\

Zustandsgleicung: $\bm{p\cdot V = n\cdot R \cdot T}$\\

Extensive Zustandsgrößen: V, N,...\\
(proportional zur Teilchenzahl)\\

Intensive Zustandsgrößen: T, p,...\\

Stoffmenge [mol]: $6,022 \cd 10^{23}$ teilchen/mol = Avogadro Konstante [$N_A$]\\

\section{Arbeit und Energie in der Thermodynamik}
$\bm{dW>0}$ Es wird Energie von außen zugeführt\\

$\bm{dW<0}$ Das System leistet Arbeit\\

\[W=-\int\limits_{V_1}^{V_2}p(V)dV\]

$W$ ist keine \textbf{Zustandsgröße}

\subsection{Wärmefluss}

\textbf{innere Energie U} (ist Zustandsgröße)\\
\[U=\frac{2}{3}k_BNT\]

\subsection{1. Hauptsatz}
 In einem abgeschlossenen System ins die innere Energie U konstant.\\
In einem nicht abgeschlossenen System gilt:
\[\Delta U = W + Q \qquad \text{ mit } W \text{ mechanische Arbeit, }Q\text{ Wärmefluss }\]

\subsection{Wärmekapazität }

\[c_v=\left.\frac{\partial U}{\partial T}\right|_V\]