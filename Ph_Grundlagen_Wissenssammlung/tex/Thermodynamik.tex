\subsection{Thermodynamik}

\textbf{Nullter Hauptsatz:} $A=C \wedge B=C \Rightarrow A=B$\\
Befinden sich zwei Systeme A und B jeweils bis Sytem C im Gleichgeweicht, befinden sich auch A und B im Gleichgewicht.\\

$\bm{p \cdot V = const.} \qquad \text{wenn } T=const$. \qquad (Boyle-Mariottesches Gesetz)\\

$\bm{V=V_0+\alpha \cdot T} \text{ für } p=const.$ \qquad (Gay Lussac)\\

1 Torr = 1 mm Hg, 1 bar = $10^5$ Pa [N/mm$^2$], Normaldruck: 1013 mbar = 1013 hPa = 760 Torr\\

Ideale Gasgleichung: $\bm{p\cdot V = N\cdot k_B \cdot T}$\\

Zustandsgleicung: $\bm{p\cdot V = n\cdot R \cdot T}$\\

Extensive Zustandsgrößen: V, N,...\\
(proportional zur Teilchenzahl)\\

Intensive Zustandsgrößen: T, p,...\\

Stoffmenge [mol]: $6,022 \cd 10^{23}$ teilchen/mol = Avogadro Konstante [$N_A$]\\



